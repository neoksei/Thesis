\chapter{Экономическая оценка проекта}
\section{Постановка задачи}
Целью дипломного проекта является
анализ стеганографических методов защиты информации.
Данный раздел содержит расчет трудоемкость и затрат на
проведение анализа предметной области, разработки и сопровождения
стеганографической системы.
\section{Оценка стоимости объектов интеллектуальной собственности}
Оценка стоимости объектов интеллектуальной собственности (ОИС),
созданных на предприятии или по его заказу (при финансировании разработок
предприятием) с закреплением за ним по договору прав собственности на них,
производится по затратному методу и определяется по формуле~\ref{eq:eco-1}:
\begin{equation}
    \label{eq:eco-1}
    C_{i} = C_\textup{р} + C_\textup{п} + C_\textup{м}
\end{equation}
где,
\begin{enumerate}
    \item $C_\textup{р}$ --- приведенные затраты на создание объектов интеллектуальной собственности, руб.;
    \item $C_\textup{п}$ --- привиденные затраты на правовую охрану объектов интеллектуальной собственности, руб.;
    \item $C_\textup{м}$ --- привиденные затраты на маркетинговые исследования ОИС, руб.
\end{enumerate}

Приведенные затраты на создание ОИС --- сумма фактически произведенных затрат
на выполнение научно-исследовательской работы (НИР) в полном объеме
(от поиска материалов исследования до формирования отчета)
и разработку всей технической документации.
Приведенные затраты для НИР состоят из затрат на поисковые работы,
включая предварительную проработку проблемы, на теоретические исследования,
на проведение экспериментов, испытаний, на услуги сторонних организаций,
на составление, рассмотрение и утверждение отчета и прочих затрат.

Приведенные затраты на разработку технической документации (ТД) состоят из затрат
на выполнение эскизного проекта, технического задания, рабочего проекта,
расчетов, испытаний, услуг сторонних организаций, авторского надзора, дизайна.
Кроме того, сюда включаются затраты на доведение ОИС до готовности
промышленного использования и коммерческой реализации.

В тех случаях, когда НИР или технологическая и проектная документация выполняется частично
или созданию ОИС предшествует проведение только НИР или разработка технической документации,
то расчет стоимости ОИС производится по затратам на фактически выполненные работы,
для товарных знаков и промышленных образцов --- затратам на дизайн.

Приведенные затраты на правовую охрану ОИС --- затраты на оформление
заявочных материалов на получение патента (свидетельства), переписка по заявке,
оплата пошлин за проведение экспертизы, получение патента (свидетельства)
и поддержание его в силе и т.д. Данная составляющая отсутствует для таких ОИС, как ноу-хау, НИР, ТД.

Приведенные затраты на маркетинговые исследования --- для целей приведения
разновременных стоимостных оценок к конечному году применяется коэффициент $\alpha_i$,
но данном случае он будет равен 1, потому что число лет, предшествующих расчетному году равно 0.
Затраты, произведенные на выполнение НИР и разработку всех стадий ТД, то есть Cp могут включать в себя:
\begin{enumerate}
    \item израсходованные материальные ресурсы;
    \item оплата труда с отчислениями разработчиков;
    \item амортизационные отчисления оборудования, которое использовалось при разработке ОИС;
    \item аренда помещения для разработчиков;
\end{enumerate}
Кроме того, создание любого ОИС или разработка программного обеспечения происходит
не всегда согласно производственного задания, где четко оговорены сроки работы.
Разработчик может вне плановых заданий выполнить(изготовить) ОИС или написать программу для ЭВМ.
В случае если отсутствует документация по затратам на создание ОИС, то их можно определить расчетным путем.

Затраты на создание ОИС в t-том году определяются по следующей формуле:
\begin{equation}
    \label{eq:eco-2}
    C_\textup{пр} = \dfrac{\textup{З}}{m}Kt
\end{equation}
Где,
\begin{enumerate}
    \item З --- среднемесячная заработная плата разработчика (разработчиков) с учетом районного коэффициента, руб.;
    \item m --- среднее количество рабочих часов в месяце;
    \item K --- коэффициент, учитывающий отчисления с заработной платы (страховые взносы).
    Ставка страховых взносов в 2021 г. составляет 30\% от величины фонда оплаты труда. K = 0,3; 
    \item t --- время в часах, затрачиваемое разработчиком (разработчиками) на создание (разработку) ОИС,
    на отладку и адаптацию ОИС к условиям производства.
\end{enumerate}

Для расчета себестоимости необходимы затраты времени.
Весь перечень произведённых работ не был оговорен рамками технического задания
с указанием конкретных сроков выполнения. Для обеспечения наибольшей достоверности
временных затрат используем метод экспертных оценок. Время работы, согласно плану
проведения аудита ИБ учтём в соответствующих этапах.

Для определения средних значений $a_\textup{i ср}$, $m_\textup{i ср}$, $b_\textup{i ср}$
используются экспертные оценки, данные руководителем и автором проекта.
Средние значения найдем по формуле\ref{eq:eco-4}.
Значения $m_i$ и $b_i$ рассчитываются аналогично.
\begin{equation}
    \label{eq:eco-3}
    a_\textup{i ср} = \dfrac{3a_\textup{i рук} + 2a_\textup{i авт}}{5}
\end{equation}
где,
\begin{enumerate}
    \item $a_\textup{i рук}$ --- оценка, данная руководителем;
    \item $a_\textup{i авт}$ --- оценка, данная автором.
\end{enumerate}

\begin{xltabular}{\textwidth}{|p{0.30\textwidth}|X|X|X|X|X|X|X|X|X|}
    \caption{Затраты времени на разработку ОИС}
    \label{tab:tabular03}
    \\ \hline
    \multirow{3}{*}{Этапы} & \multicolumn{9}{c|}{Величина затрат} \\ \cline{2-10} 
     & \multicolumn{3}{l|}{Минимальная} & \multicolumn{3}{l|}{Вероятная} & \multicolumn{3}{l|}{Максимальная} \\ \cline{2-10} 
     & \rotatebox[origin=c]{90}{ Руководитель } & \rotatebox[origin=c]{90}{ Автор } & \rotatebox[origin=c]{90}{ Средняя } & \rotatebox[origin=c]{90}{ Руководитель } & \rotatebox[origin=c]{90}{ Автор } & \rotatebox[origin=c]{90}{ Средняя } & \rotatebox[origin=c]{90}{ Руководитель } & \rotatebox[origin=c]{90}{ Автор } & \rotatebox[origin=c]{90}{ Средняя } \\ \hline
    Ознакомление с исходными данными & 3 & 4 & 3.5 & 4 & 6 & 5 & 5 & 7 & 6 \\ \hline
    Анализ предметной области & 9 & 17 & 13 & 13 & 25 & 18 & 25 & 41 & 33 \\ \hline
    Разработка системы & 179 & 210 & 190 & 250 & 310 & 280 & 330 & 410 & 370 \\ \hline
    Вывод в эксплуатацию & 9 & 17 & 13 & 17 & 31 & 24 & 25 & 41 & 33 \\ \hline
\end{xltabular}

Ожидаемая величина затрат для i-го этапа ($MO_i$) и стандартное отклонение этой величины
каждого i-го этапа ($G_i$):
\begin{equation}
    \label{eq:eco-4}
    MO_i = \dfrac{\alpha_i + 4m_i + b_i}{6}
\end{equation}
\begin{equation}
    \label{eq:eco-5}
    G_i = \dfrac{b_i - a_i}{6}
\end{equation}
Результаты показаны в таблице~\ref{tbl:eco-2}

\newcolumntype{L}[1]{>{\raggedright\arraybackslash}p{#1}}

\begin{xltabular}{\textwidth}{|L{0.3\linewidth}|L{0.1\linewidth}|L{0.1\linewidth}|L{0.1\linewidth}|X|X|}
    \caption{Затраты времени на разработку ОИС}
    \label{tbl:eco-2}
    \\
    \hline
        \multirow{2}{*}{\parbox{1.0\linewidth}{Этапы разработки ОИС}} &
        \multicolumn{3}{L{0.3\linewidth}|}{Средняя величина затрат времени этапа разработки ОИС} &
        \multirow{2}{*}{$MO_i$} &
        \multirow{2}{*}{$G_i$} \\
    \cline{2-4}
        &
        $a_i$ &
        $m_i$ &
        $b_i$ &
        & \\
    \hline
        Ознакомление с исходными данными &
        3.5 &
        5 &
        6 &
        4.9 &
        0.4 \\
    \hline
        Анализ предметной области &
        13 &
        18 &
        33 &
        19.6 &
        3.3 \\
    \hline
        Разработка системы &
        190 &
        280 &
        370 &
        280 &
        30 \\
    \hline
        Вывод системы в опытную эксплуатацию &
        13 &
        24 &
        33 &
        23.6 &
        3.3 \\
    \hline
\end{xltabular}
Зная ожидаемые затраты и стандартное отклонение по каждому этапу,
рассчитываются эти показатели в целом по ОИС:
\begin{equation}
    MO = \sum_{i=1}^{n}MO_i
\end{equation}
\begin{equation}
    G = \sqrt{\sum_{i=1}^{n}G_i^2}
\end{equation}

\begin{align*}
    MO = 4.9 + 19.6 + 280 + 23.6 = 328.1
\end{align*}
\begin{align*}
    G = \sqrt{0.4^2 + 3.3^2 + 30^2 + 3.3^2} = 30.4
\end{align*}
\section{Оценка стоимости разработки}
Необходимо определить себестоимость разработки системы.
Стоимость разработки ОИС найдём по формуле~\ref{eq:eco-2} при следующих данных:
\begin{enumerate}
    \item среднемесячная заработная плата разработчика с учетом коэффициента составляет 90 000 руб.;
    \item среднее количество рабочих часов в месяце --- 168;
    \item затраты времени на разработку ОИС: $328.1 + 30 = 358.1$ часа.
\end{enumerate}

\begin{align*}
    \textup{С} = \dfrac{90000}{168}*358.1 = 191839.3
\end{align*}

Прочие расходы:
\begin{enumerate}
    \item Ставка страховых взносов в 2021 г. составляет 30\% от величины фонда оплаты труда.
    В нашем примере они составят:
    Страховые взносы = $C*0.3 = 57551.7$
\end{enumerate}

Общие затраты на разработку составят:
\begin{align*}
    \textup{З} = \textup{З}_\textup{прям} + \textup{З}_\textup{пр} = 191839.3 + 57551.1 = 249390.4
\end{align*}

\section{Оценка стоимости использования оборудования и сопровождения системы}
Для развертывания системы будет использоваться выделенный сервер.
Траты для сервера с конфигурацией: четырехядерный процессор с восмью потоками, 16 гигабайт оперативной
памяти, 1 терабайт SSD --- 6 000 рублей/месяц. Сопровождением системы займется системный администратор.
По формуле выше найдем затраты на сотрудника при учете месячного оклада 40 000 рублей:
\begin{align*}
    \textup{С} = 40000 + 40000 * 0.3 =  52000 \textup{руб.}
\end{align*}
Суммарные среднемесячные затраты на систему составляют 58000 рублей.

\section{Экономическое обоснование проекта}
Стоимость разработки проекта составляет 249 390.4 рублей.
Анализ показал, что данная система позволит сократить убытки
компании от несанкционированного копирования и распространения цифровых объектов
в 2 раза. В среднем компания теряет 240 000 рублей в месяц из-за несанкционированного
копирования и распространения цифровых объектов компании. Рассчитаем, сколько проект приносит прибыли в месяц:
\begin{align*}
    \textup{P} = \dfrac{240000}{2} - 58000 = 62000
\end{align*}
где P --- это прибыль.
Итого срок окупаемости проекта составляет пять месяцев.