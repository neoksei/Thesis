\Introduction
Стеганография --- это практика сокрытия сообщения внутри другого сообщения или физического объекта.
Тогда как криптография скрывает содержимое сообщение, стеганография скрывает сам факт существования какого-либо сообщения. 

Стеганография часто используется совместно с криптографией, дополняя ее.
Стеганографические методы сокрытия сообщения снижают вероятность обнаружения самой передачи сообщения.
Если сообщение к тому же зашифровано, то это обеспечивает еще большую защищенность. 

В настоящее время наибольшее распространение получила цифровая стеганография.
Особенностью цифровой стеганографии является сокрытие информации внутри других цифровых объектов,
таких как текст, изображения, видео, аудио и другие.
Со все большим возрастанием роли интернет-технологий в жизни человека значимость стеганографии также возрастает.

Области применения стеганографии включают в себя:
\begin{enumerate}
    \item Защита авторского права и DRM (digital rights management).
    \item Незаметная передача информации.
    \item Защита конфиденциальной информации от несанкционированного доступа.
\end{enumerate}

Цифровая стеганография является молодым и бурно развивающимся направлением.
Несмотря на свою востребованность, стеганография очень слабо освещается в программах обучения информационной безопасности.
Моя работа призвана восполнить этот пробел.

В своей работе я хочу рассмотреть основные стеганографические алгоритмы, определить области их применения,
достоинства и недостатки, а также провести анализ возможных уязвимостей этих алгоритмов.
