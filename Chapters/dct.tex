\chapter{Сокрытие в спектральной области}
\section{Общие сведения}

\section{Дискретное косинусное преобразование}
Дискретное косинусное преобразование (ДКП) --- одно из дискретных преобразований Фурье.
ДКП представляет конечную последовательность в виде суммы функций косинуса,
колеблющихся на разных частотах. ДКП широко используется при обработке сигналов и сжатии данных.
Например, ДКП используется при сжатии в изображениях (JPEG, HEIF), аудиофайлах (Dolby Digital, MP3),
видеофайлах (MPEG, H.26x), в цифровом телевидении (SDTV, HDTV, VOD) и в других.

ДКП является линейным ортогональным преобразованием. Как любое дискретное линейное преобразование,
ДКП можно представить в виде матрицы. Буду ортоганальным преобразованием, обратное к ДКП преобразование
задает транспонированной матрицей ДКП, домноженной на какой-то коэффициент.

Использование косинусных, а не синусоидальных функций имеет решающее значение для сжатия,
поскольку для аппроксимации типичного сигнала требуется меньше косинусных функций.
ДКП подобно дискретному преобразованию Фурье, но использующее только действительные числа.

Существует 8 стандартных типов ДКП, однако наиболее употребимым является второй тип,
который часто называют просто ДКП (DCT-II).
Формула дискретного косинусного преобразования выглядит так,
как показано в формуле~\ref{eq:simple-dcp}:
\begin{equation} \label{eq:simple-dcp}
    X_k = \sum_{n=0}^{N-1} x_n \cos \left[\frac{\pi}{N} \left(n+\frac{1}{2}\right) k \right] \quad \quad k = 0, \dots, N-1    
\end{equation}
Формула для матрицы преобразования выглядит как формуле~\ref{eq:matrix-dcp}:
\begin{equation} \label{eq:matrix-dcp}
    {DCT}\text{-}2_n= \left[\cos (k(l+\tfrac{1}{2})\tfrac{\pi}{n})\right]_{0\leq k,l<n}    
\end{equation}

Как и в случае быстрого преобразования Фурье, существуют алгоритмы быстрого ДКП преобразования.

DCT-II часто используется для сжатия с потерями благодаря своему свойству уплотнения энерегии:
в типичных случаях большая часть информации, которую содержит сигнал, концентрируется в нескольких
первых коэффициентах разложения.

Существуют так же многомерные ДКП, которые получаются из одномерных путем композиции ДКП по каждому измерению.
Вывод такого преобразования для двумерного случая показан в формуле~\ref{eq:2d-dcp}.
\begin{align}
    X_{k_1,k_2} &= \nonumber
    \sum_{n_1=0}^{N_1-1}
    \left( \sum_{n_2=0}^{N_2-1}
    x_{n_1,n_2} 
    \cos \left[\frac{\pi}{N_2} \left(n_2+\frac{1}{2}\right) k_2 \right]\right)
    \cos \left[\frac{\pi}{N_1} \left(n_1+\frac{1}{2}\right) k_1 \right]\\
    &= \sum_{n_1=0}^{N_1-1}
    \sum_{n_2=0}^{N_2-1}
    x_{n_1,n_2} 
    \cos \left[\frac{\pi}{N_1} \left(n_1+\frac{1}{2}\right) k_1 \right]
    \cos \left[\frac{\pi}{N_2} \left(n_2+\frac{1}{2}\right) k_2 \right] \label{eq:2d-dcp}
\end{align}
Здесь $[x_{n_1,n_2}]$ --- матрица до преобразования, и $[X_{k_1,k_2}]$ --- матрица
после преобразования.
В матричном виде это преобразование может быть представлено так, как показано
в формуле~\ref{eq:2d-matrix-dcp}, где $x$ --- матрица, которую нужно преобразовать.
\begin{equation} \label{eq:2d-matrix-dcp}
    X = ({DCT}\text{-}2_n) x ({DCT}\text{-}2_n ^ T)
\end{equation}
Именно такое преобразование используется при компрессии в JPEG.

\section{JPEG}
JPEG является широко используемым методом сжатия с потерями для цифровых изображений.
Степень сжатия регулируется, что позволяет выбирать между качеством и размером изображения.
JPEG наиболее широко используемый стандарт сжатия изображений в мире и
наиболее используемый формат цифровых изображений.

ДКП лежит в основе сжатия методом JPEG. Как уже говорилось выше,
ДКП был выбран именно благодаря свойству уплотнения энергии. Чтобы прояснить,
о чем идет речь, мной была сделана визуализация преобразования ДКП.

Выберем на изображении область 32x32 пикселя, как показано на рисунке~\ref{img:lenna-eye}.

\begin{figure}[ht!]
    \centering
    \includegraphics[width=\linewidth]{DCT/Lenna_eye.png}
    \caption{Выбираем область}
    \label{img:lenna-eye}
\end{figure}

%Для простоты будем использовать только синий канал изображения.
Сначала рассмотрим матрицу пикселей как двумерную дискретную функцию.
Расположим координаты так, чтобы в левом верхнем углу
располагался пиксель с координатами $p_{kj}, k = 0, j = 0$.
Визуализацию можно посмотреть на рисунке~\ref{img:pixels-dct}.

\begin{figure}[ht!]
    \centering
    \caption{Визуализация пикселей}
    \includegraphics[width=\linewidth]{DCT/pixels.png}
    \label{img:pixels-dct}
\end{figure}

Умножим эту функцию справа на транспонированную матрицу ДКП
по формуле~\ref{eq:2d-matrix-dcp}. Мы получим новую функцию,
которая показана на рисунке~\ref{img:dct-1}.
Таким образом фактически ДКП применилось к каждой строке матрицы.
Из изображения видно, что наибольшие коэффициенты расположены
в нижней части спектра, то есть ближе к нулевому столбцу.

\begin{figure}[ht!]
    \centering
    \caption{После приминения ДКП к строкам матрицы}
    \includegraphics[width=\linewidth]{DCT/dct-1.png}
    \label{img:dct-1}
\end{figure}

К полученной матрице применим ДКП еще раз, в этот раз по столбцам.
В полученной матрице наибольшее значение имеет коэфициент с координатами
$k = 0, j = 0$. Этот коэффициент называется DC-коэффициент.
Остальные коэффициенты называются AC-коэффициентами.
Матрица показана на рисунке~\ref{img:dct-2}.

\begin{figure}[ht!]
    \centering
    \caption{После приминения ДКП к строкам матрицы}
    \includegraphics[width=\linewidth]{DCT/dct-2.png}
    \label{img:dct-2}
\end{figure}

DC-коэфициент блока равен среднему всех пикселей в блоке,
взятому с определенным коэффициентом. Удаляя все коэффициенты,
кроме DC, мы можем аппроксимировать блок пикселей их средним
арифметическим. Чем дальше коэффициент располагается от DC,
тем меньше психовизуальной информации он несет для человека,
и тем более незаметные детали изображения он хранит в себе.
Соответственно, основная идея алгоритма состоит в отбрасывании
наименее значимых коэффициентов. Это позволяет производить сжатие
изображения с потерями.

Алгоритм сжатия JPEG работает с каждым каналом отдельно, поэтому для
простоты рассмотрим работу JPEG на изображении в режиме градации серого.
В самом начале своей работы алгоритм разбивает изображение на блоки
8x8 пикселей. К каждому блоку применяется ДКП преобразование,
что равносильно разложению исходной матрицы по базису, состоящему
из 64 функций. Эти 64 функции показаны на рисунке~\ref{img:basis}.
Из этого рисунка видно, что помере отдаления от левого верхнего которая
функции становятся все более рельефными, что объясняет, почему они несут
наиболее мелкие детали изображения. Так же видно, что функция,
соответствующая DC-коэффициенту, представленая плоскостью. Очевидно,
что лучшим константным приближением функции
является ее математическое ожидание.
\begin{figure}[ht!]
    \centering
    \caption{Базис ДКП}
    \includegraphics[width=\linewidth]{DCT/basis.png}
    \label{img:basis}
\end{figure}
После приминения ДКП преобразования к блоку матрице 8x8
получается другая матрица той же размерности. В соотвтствии с вышесказанным
эта матрица делится на области низких, средних и высоких частот. В таком порядке
убывает информативность коэффициентов. Это можно увидеть на рисунке~\ref{img:freq}
\begin{figure}[ht!]
    \centering
    \caption{После приминения ДКП к строкам матрицы}
    \includegraphics[width=\linewidth]{DCT/freq.png}
    \label{img:freq}
\end{figure}