\chapter{Общие положения и описание стеганографических методов защиты информации} % Общие положения и описание стеганографических методов защиты информации

\section{Основные понятия}

\begin{enumerate}
    \item Стеганографическая система (стегосистема) --- совокупность методов и средств,
    используемых для создания скрытого канала для передачи информации.
    Основные требования, предъявляемые к стегосистеме:
    \begin{enumerate}
        \item Безопасность системы определяется секретностью ключа.
        Это означает, что даже если потенциальный враг представляет работу стеганографической системы
        и статистические характеристики сообщений и контейнеров,
        это не дает ему дополнительных преимуществ при выявлении
        наличия или отсутствия сообщения в конкретном контейнере.
        \item При обнаружении противником наличия скрытого сообщения
        он не должен смочь извлечь сообщение до тех пор, пока он не будет владеть ключом.
        \item Алгоритм сокрытия информации не нарушает ее целостность и аутентичность.
        В некоторых случаях дополнительно требуется,
        чтобы алгоритм обеспечивал целостность сообщения при деформации контейнера.
        \item Система с цифровым водяным знаком должна иметь низкую вероятность ложного обнаружения скрытого сообщения.
    \end{enumerate}
    \item Сообщение --- информация, передачу которой нужно скрыть.
    \item Контейнер --- любая информация,
    используемая для сокрытия тайного сообщения. Контейнер может находиться в одном из двух состояний:
    \begin{enumerate}
        \item Пустой контейнер --- контейнер, не содержащий сообщение.
        \item Заполненный контейнер (стегоконтейнер) --- контейнер, содержащий сообщение.
    \end{enumerate}
    \item Стеганографический канал (стегоканал) --- канал передачи стегоконтейнера.
    \item Ключ (стегоключ) --- секретный ключ, нужный для сокрытия стегоконтейнера.
    По аналогии с криптографией стегоключи подразделяются на 2 типа:
    \begin{enumerate}
        \item Закрытый стегоключ.
        В системах с закрытым стегоключем ключ должен быть создан до начала обмена сообщениями,
        либо передан по защищенному каналу связи.
        \item Открытый стегоключ. Такой ключ может быть передан по открытому незащищенному каналу связи.
        Открытый стегоключ должен обладать таким свойством, чтобы по нему вычислительно нецелесообразно
        было восстанавливать закрытый ключ.
    \end{enumerate}
\end{enumerate}

\section{Цифровые водяные знаки и цифровые отпечатки}

Цифровой водяной знак (ЦВЗ) --- технология, созданная для защиты авторский прав на цифровые объекты.
В связи с быстрым развитием информационных технологий все более актуальным становится
вопрос защиты авторских прав и интеллектуальной собственности, представленной в цифровом виде.
Примерами цифровых объектов могут выступать аудиозаписи, видеозаписи, изображения, электронные книги и другие. 
ЦВЗ могут быть как видимыми, так и невидимыми. Решение о наличии в цифровом объекте невидимого ЦВЗ принимаются
на основе процедуры декодирования.

В общем виде стегосистема ЦВЗ может быть разбита на части следующим образом:
\begin{enumerate}
    \item Прекодер --- часть, которая приводит ЦВЗ к удобному для встраивания в стегоконтейнер виду.
    \item Стегокодер --- часть, которая вкладывает сообщение в стегоконтейнер.
    \item Выделение встроенного сообщения --- процедура, выделяющие сообщение из стегоконтейнера.
    \item Стегодетектор --- часть, определяющая наличие ЦВЗ.
    \item Декодер --- часть, восстанавливающая исходное сообщение.
\end{enumerate}

Контейнер, содержащий ЦВЗ, может подвергаться преднамеренным атакам или случайным помехам.
Стегосистема ЦВЗ должна обеспечивать как различимость самого стегоконтейнера человеком,
потому как в качестве стегоконтейнера выступает интеллектуальная собственность,
нацеленная на конечного потребителя, так и различимость ЦВЗ стегодетектором,
который может подтвердить или опровергнуть авторские права на интеллектуальную собственность.
В связи с этим в стегосистемах ЦВЗ применяется помехоустойчивое кодирование и метод широкополосного сигнала.

Одной из основных характеристик ЦВЗ является надежность.
Под надежность понимается устойчивость к различным деформациям контейнера.
По отношению к этой характеристика ЦВЗ распадается на три класса:
\begin{enumerate}
    \item Хрупкие.
    Такие ЦВЗ разрушается при небольших модификациях заполненного стегоконтейнера.
    Такие ЦВЗ применяются для аутентификации сигнала.
    Например, такие ЦВЗ используются для подтверждения подлинности цифрового объекта.
    \item Полухрупкие.
    Такие ЦВЗ чувствительны к некоторым преобразованиям контейнера и нечувствительны к другим.
    Например, ЦВЗ, встроенное в изображение, может быть нечувствительно к его компрессии,
    но в то же время быть чувствительно к вырезке из этого изображения фрагмента.
    \item Робастные или надежные.
    Такие ЦВЗ устойчивы к разным видам воздействия на контейнер.
    Такие ЦВЗ часто применяются при защите от копирования.
\end{enumerate}

Чтобы осуществить вложение ЦВЗ в стегоконтейнер, ЦВЗ преобразуют к более удобному виду.
Например, если ЦВЗ является изображением, то удобно будет представить его как двумерную
битовую матрицу. Так же если ЦВЗ является изображением, разумно будет использовать не само
изображение, а его Вайвлет преобразование или дискретное косинусное преобразование. 
Изображения обладают большой визуальной избыточностью.
Данные преобразования концентрируют большую часть энергии (визуальной информации) в нижних частотах.
Поэтому их можно использовать как низкочастотные фильтры. То же самое относится и к контейнерам.

Похожим на ЦВЗ, но отличающимся понятием является цифровой отпечаток.
ЦВЗ предполагает встраивание одного и того же сообщения в различные контейнеры.
В случае же цифрового отпечатка в каждый контейнер встраивается уникальное сообщение.
Часто областью применения цифровых отпечатков становится защита исключительного права.
В качестве сообщения в таком случае встраивается информация, указывающая на идентифицирующие данные покупателя.
Эти данные позволят отследить источник распространения, если произойдет утечка стегоконтейнера.

\section{Обобщенные стеганографические методы}
К настоящему моменту разработано множество стеганографических методов скрытия информации.
Для их систематичного изучения удобно группировать их по схожим признакам.
\begin{enumerate}
    \item Пространственные методы. Особенностью этих методов является сокрытие информации напрямую
    в пространственной области контейнера. Например,
    в случае звукового контейнера  таким пространством могут быть семплы,
    а в случае изображения --- пиксели.
    \item Частотные методы. Такие методы сначала используют одно из интегральный преобразований сигнала,
    чтобы перейти в его частотную область. Далее кодирование сообщение производится за счет изменения частотных
    характеристик сигнала. После этого используется обратное преобразование, чтобы получить модифицированный сигнал,
    содержащий закодированное сообщение.
    \item Алгоритмы, использующие особенности формата файла. Такие алгоритмы как правило записывают
    сообщение в метаданные файла или в иные неиспользуемые поля файла.
\end{enumerate}

\section{Стегоанализ}
Стегоанализ --- это наука о выявлении сообщений, скрытых методами стеганографии.
Задача стегоанализа --- выявить подозрительные контейнеры, определить, есть ли в них скрытое сообщение,
и, если возможно, восстановить это сообщение.

Если в случае криптоанализа аналитик начинает работу сразу с зашифрованным сообщением,
то в случае стегоанализа аналитик начинает работу с множества подозрительных файлов,
о которых как правило мало что известно. Деятельность аналитика в таком случае
начинается с сокращения этого множества файлов до подмножества, в котором файлы
скорее всего были заполнены сообщением.

Самой простым методом стегоанализа является субъективная атака. Атака заключается
в попытке "на глаз" определить, содержит тот или иной контейнер стего. Несмотря
на свою простоту, атака часто применяется на начальном этапе стегоанализа системы.

Основной техникой, используемой в стегоанализе, является статистический анализ.
Сначала множество незаполненных контейнеров одного типа анализируется для получения
различной статистики. Затем эта статистка используется при классификации контейнера
как заполненного или пустого. При такой классификации могут быть использованы
самые различные наблюдения:
\begin{enumerate}
    \item Сокрытие информации может приводить к изменению статистической структуры
    контейнера, в результате чего соседние элементы контейнера становятся попарно ближе
    друг к другу. На этом основана атака хи-квадрат.
    \item Сокрытие информации увеличивает энтропию контейнера. В результате чего он
    хуже поддается сжатию. На этом основана атака с помощью алгоритмов сжатия.
    \item Различные статистические данные контейнера и его областей можно использовать
    как вектор признаков. Собрав большой датасет таких признаковых описаний объектов стего,
    на нем можно обучить нейросеть, которая будет классифицировать изображения.
\end{enumerate}.