% 1. Стиль и язык
\documentclass[utf8x, 14pt]{G7-32} % Стиль (по умолчанию будет 14pt)
% Остальные стандартные настройки убраны в preamble.inc.tex.
\include{preamble.inc}
% Настройки листингов.
% \ifPDFTeX
\include{listings.inc}
% \else
% \usepackage{local-minted}
% \fi

% Полезные макросы листингов.
\include{macros.inc}

\begin{document}

\frontmatter % выключает нумерацию ВСЕГО; здесь начинаются ненумерованные главы: реферат, введение, глоссарий, сокращения и прочее.


%\listoffigures                         % Список рисунков

%\listoftables                          % Список таблиц

%\NormRefs % Нормативные ссылки 
% Команды \breakingbeforechapters и \nonbreakingbeforechapters
% управляют разрывом страницы перед главами.
% По-умолчанию страница разрывается.

% \nobreakingbeforechapters
% \breakingbeforechapters

\tableofcontents

% \printnomenclature % Автоматический список сокращений
\Abbreviations
\begin{description}
    \item[CRC] cyclic redundancy code (циклический избыточный код)
    \item[LSB] least significant bit (наименее значащий бит)
    \item[DRM] digital rights management (цифровое управление правами)
    \item[ЦВЗ] цифровой водяной знак
    \item[RGB] red, green, blue (красный, синий, зеленый)
    \item[ДКП] дискретное косинусное преобразование
    \item[DCT] discrete cosine transform (дискретное косинусное преобразование)
    \item[НИР] научно-исследовательская работа
    \item[ОИС] объект интеллектуальной собственности
    \item[ГПСЧ] генератор псевдослучайных чисел 
\end{description}
\Introduction
Стеганография --- это практика сокрытия сообщения внутри другого сообщения или физического объекта.
Тогда как криптография скрывает содержимое сообщение, стеганография скрывает сам факт существования какого-либо сообщения.

Стеганография часто используется совместно с криптографией, дополняя ее.
Стеганографические методы сокрытия сообщения снижают вероятность обнаружения самой передачи сообщения.
Если сообщение к тому же зашифровано, то это обеспечивает еще большую защищенность. 

В настоящее время наибольшее распространение получила цифровая стеганография.
Особенностью цифровой стеганографии является сокрытие информации внутри других цифровых объектов,
таких как текст, изображения, видео, аудио и другие.
Со все большим возрастанием роли интернет-технологий в жизни человека значимость стеганографии также возрастает.

Области применения стеганографии включают в себя:
\begin{enumerate}
    \item Защита авторского права и DRM (digital rights management).
    \item Незаметная передача информации.
    \item Защита конфиденциальной информации от несанкционированного доступа.
\end{enumerate}

Цифровая стеганография является молодым и бурно развивающимся направлением. В последние годы
стеганография все больше находит приминение в области защиты прав собственности на информацию.

В своей работе я хочу рассмотреть основные стеганографические алгоритмы, определить области их применения,
достоинства и недостатки, а также провести анализ возможных уязвимостей этих алгоритмов.



\mainmatter % это включает нумерацию глав и секций в документе ниже

\chapter{Общие положения и описание стеганографических методов защиты информации} % Общие положения и описание стеганографических методов защиты информации

\section{Основные понятия}

\begin{enumerate}
    \item Стеганографическая система (стегосистема) --- совокупность методов и средств,
    используемых для создания скрытого канала для передачи информации.
    Основные требования, предъявляемые к стегосистеме:
    \begin{enumerate}
        \item Безопасность системы определяется секретностью ключа.
        Это означает, что даже если потенциальный враг представляет работу стеганографической системы
        и статистические характеристики сообщений и контейнеров,
        это не дает ему дополнительных преимуществ при выявлении
        наличия или отсутствия сообщения в конкретном контейнере.
        \item При обнаружении противником наличия скрытого сообщения
        он не должен смочь извлечь сообщение до тех пор, пока он не будет владеть ключом.
        \item Алгоритм сокрытия информации не нарушает ее целостность и аутентичность.
        В некоторых случаях дополнительно требуется,
        чтобы алгоритм обеспечивал целостность сообщения при деформации контейнера.
        \item Система с цифровым водяным знаком должна иметь низкую вероятность ложного обнаружения скрытого сообщения.
    \end{enumerate}
    \item Сообщение --- информация, передачу которой нужно скрыть.
    \item Контейнер --- любая информация,
    используемая для сокрытия тайного сообщения. Контейнер может находиться в одном из двух состояний:
    \begin{enumerate}
        \item Пустой контейнер --- контейнер, не содержащий сообщение.
        \item Заполненный контейнер (стегоконтейнер) --- контейнер, содержащий сообщение.
    \end{enumerate}
    \item Стеганографический канал (стегоканал) --- канал передачи стегоконтейнера.
    \item Ключ (стегоключ) --- секретный ключ, нужный для сокрытия стегоконтейнера.
    По аналогии с криптографией стегоключи подразделяются на 2 типа:
    \begin{enumerate}
        \item Закрытый стегоключ.
        В системах с закрытым стегоключем ключ должен быть создан до начала обмена сообщениями,
        либо передан по защищенному каналу связи.
        \item Открытый стегоключ. Такой ключ может быть передан по открытому незащищенному каналу связи.
        Открытый стегоключ должен обладать таким свойством, чтобы по нему вычислительно нецелесообразно
        было восстанавливать закрытый ключ.
    \end{enumerate}
\end{enumerate}

\section{Цифровые водяные знаки и цифровые отпечатки}

Цифровой водяной знак (ЦВЗ) --- технология, созданная для защиты авторский прав на цифровые объекты.
В связи с быстрым развитием информационных технологий все более актуальным становится
вопрос защиты авторских прав и интеллектуальной собственности, представленной в цифровом виде.
Примерами цифровых объектов могут выступать аудиозаписи, видеозаписи, изображения, электронные книги и другие. 
ЦВЗ могут быть как видимыми, так и невидимыми. Решение о наличии в цифровом объекте невидимого ЦВЗ принимаются
на основе процедуры декодирования.

В общем виде стегосистема ЦВЗ может быть разбита на части следующим образом:
\begin{enumerate}
    \item Прекодер --- часть, которая приводит ЦВЗ к удобному для встраивания в стегоконтейнер виду.
    \item Стегокодер --- часть, которая вкладывает сообщение в стегоконтейнер.
    \item Выделение встроенного сообщения --- процедура, выделяющие сообщение из стегоконтейнера.
    \item Стегодетектор --- часть, определяющая наличие ЦВЗ.
    \item Декодер --- часть, восстанавливающая исходное сообщение.
\end{enumerate}

Контейнер, содержащий ЦВЗ, может подвергаться преднамеренным атакам или случайным помехам.
Стегосистема ЦВЗ должна обеспечивать как различимость самого стегоконтейнера человеком,
потому как в качестве стегоконтейнера выступает интеллектуальная собственность,
нацеленная на конечного потребителя, так и различимость ЦВЗ стегодетектором,
который может подтвердить или опровергнуть авторские права на интеллектуальную собственность.
В связи с этим в стегосистемах ЦВЗ применяется помехоустойчивое кодирование и метод широкополосного сигнала.

Одной из основных характеристик ЦВЗ является надежность.
Под надежность понимается устойчивость к различным деформациям контейнера.
По отношению к этой характеристика ЦВЗ распадается на три класса:
\begin{enumerate}
    \item Хрупкие.
    Такие ЦВЗ разрушается при небольших модификациях заполненного стегоконтейнера.
    Такие ЦВЗ применяются для аутентификации сигнала.
    Например, такие ЦВЗ используются для подтверждения подлинности цифрового объекта.
    \item Полухрупкие.
    Такие ЦВЗ чувствительны к некоторым преобразованиям контейнера и нечувствительны к другим.
    Например, ЦВЗ, встроенное в изображение, может быть нечувствительно к его компрессии,
    но в то же время быть чувствительно к вырезке из этого изображения фрагмента.
    \item Робастные или надежные.
    Такие ЦВЗ устойчивы к разным видам воздействия на контейнер.
    Такие ЦВЗ часто применяются при защите от копирования.
\end{enumerate}

Чтобы осуществить вложение ЦВЗ в стегоконтейнер, ЦВЗ преобразуют к более удобному виду.
Например, если ЦВЗ является изображением, то удобно будет представить его как двумерную
битовую матрицу. Так же если ЦВЗ является изображением, разумно будет использовать не само
изображение, а его Вайвлет преобразование или дискретное косинусное преобразование. 
Изображения обладают большой визуальной избыточностью.
Данные преобразования концентрируют большую часть энергии (визуальной информации) в нижних частотах.
Поэтому их можно использовать как низкочастотные фильтры. То же самое относится и к контейнерам.

Похожим на ЦВЗ, но отличающимся понятием является цифровой отпечаток.
ЦВЗ предполагает встраивание одного и того же сообщения в различные контейнеры.
В случае же цифрового отпечатка в каждый контейнер встраивается уникальное сообщение.
Часто областью применения цифровых отпечатков становится защита исключительного права.
В качестве сообщения в таком случае встраивается информация, указывающая на идентифицирующие данные покупателя.
Эти данные позволят отследить источник распространения, если произойдет утечка стегоконтейнера.

\section{Обобщенные стеганографические методы}
К настоящему моменту разработано множество стеганографических методов скрытия информации.
Для их систематичного изучения удобно группировать их по схожим признакам.
\begin{enumerate}
    \item Пространственные методы. Особенностью этих методов является сокрытие информации напрямую
    в пространственной области контейнера. Например,
    в случае звукового контейнера  таким пространством могут быть семплы,
    а в случае изображения --- пиксели.
    \item Частотные методы. Такие методы сначала используют одно из интегральный преобразований сигнала,
    чтобы перейти в его частотную область. Далее кодирование сообщение производится за счет изменения частотных
    характеристик сигнала. После этого используется обратное преобразование, чтобы получить модифицированный сигнал,
    содержащий закодированное сообщение.
    \item Алгоритмы, использующие особенности формата файла. Такие алгоритмы как правило записывают
    сообщение в метаданные файла или в иные неиспользуемые поля файла.
\end{enumerate}

\section{Стегоанализ}
Стегоанализ --- это наука о выявлении сообщений, скрытых методами стеганографии.
Задача стегоанализа --- выявить подозрительные контейнеры, определить, есть ли в них скрытое сообщение,
и, если возможно, восстановить это сообщение.

Если в случае криптоанализа аналитик начинает работу сразу с зашифрованным сообщением,
то в случае стегоанализа аналитик начинает работу с множества подозрительных файлов,
о которых как правило мало что известно. Деятельность аналитика в таком случае
начинается с сокращения этого множества файлов до подмножества, в котором файлы
скорее всего были заполнены сообщением.

Самой простым методом стегоанализа является субъективная атака. Атака заключается
в попытке ``на глаз'' определить, содержит тот или иной контейнер стего. Несмотря
на свою простоту, атака часто применяется на начальном этапе стегоанализа системы.

Основной техникой, используемой в стегоанализе, является статистический анализ.
Сначала множество незаполненных контейнеров одного типа анализируется для получения
различной статистики. Затем эта статистка используется при классификации контейнера
как заполненного или пустого. При такой классификации могут быть использованы
самые различные наблюдения:
\begin{enumerate}
    \item Сокрытие информации может приводить к изменению статистической структуры
    контейнера, в результате чего соседние элементы контейнера становятся попарно ближе
    друг к другу. На этом основана атака хи-квадрат.
    \item Сокрытие информации увеличивает энтропию контейнера. В результате чего он
    хуже поддается сжатию. На этом основана атака с помощью алгоритмов сжатия.
    \item Различные статистические данные контейнера и его областей можно использовать
    как вектор признаков. Собрав большой датасет таких признаковых описаний объектов стего,
    на нем можно обучить нейросеть, которая будет классифицировать изображения.
\end{enumerate}.
\chapter{LSB}
\section{Общие сведения}
LSB (least significant bit) --- стеганографический метод сокрытия информации, основанный на замене поседних значащих бит
элементов контейнера битами сообщения. Этот метод использует тот факт,
что уровень детализации во многих контейнерах гораздо выше того,
что может воспринять и различить человек.
Следовательно, заполенный контейнер будет неотличим от оригинального
для человеческого восприятия. В качестве примера можно взять
полутоновое изображение с градациями серого. Цвет кодируется одним байтом.
Человеческий глаз воспринимает только первые 7 байт,
а самый младший бит вносит там мало информации, что человек не сможет заметить разницу.

LSB обладает следующими достоинствами:
\begin{enumerate}
    \item Простота реализации и эффективность.
    \item Низкая вычислительная сложность.
    \item Пустой и заполненный контейнер неразличимы для органов восприятия человека
\end{enumerate}
И недостатками:
\begin{enumerate}
    \item Метод применим лишь к контейнерам, которые хранят данные без сжатия или используют
    сжатие без потерь, так как информации, закодированная в наименее значимых битах, может
    быть потеряна в процессе сжатия.
    \item Небольшие трансформации контейнера приводят к невозможности восстановить сообщение.
    Например, если сообщение скрыто в изображении методом LSB, то небольшие линейные трансформации
    (вращение, движение, отражение, гомотетия, сжатие, растяжение) уничтожат сообщение. Так же
    сообщение разрушается в результате сжатия с потерями. Все это говорит о том, что метод обладает
    низкой робастостью.
    \item Факт сокрытия изображения легко обнаруживает методами стегоанализа.
\end{enumerate}

Ввиду перечисленных выше недостатков очевидным кажется недопустимость использования
данного методы для сокрытия ЦВЗ.

\section{Алгоритм}
Перейдем к конкретным реализациям этого метода. Алгоритм~\ref{alg:lsb_encode}
демонстрирует псевдокод сокрытия методом LSB.
\begin{algorithm}[ht!]
    \KwData{Контейнер, Сообщение}
    \KwResult{Заполненный стегоконтейнер}
     $N \leftarrow$ Длина сообщения в битах\;
     $Message \leftarrow$ Бинарное представление сообщения\;
     $Container \leftarrow$ Массив с элементами контейнера\;
     \For{$i = 1, 2, \ldots, N$}{
         \eIf{$Container[i] \equiv Message[i] \pmod{2}$}{
             \textbf{continue\;}
         }{
             $Container[i] \leftarrow (Container[i] \land \neg 1) \lor Message[i]$\;
         }
     }
     \caption{LSB Кодирование}
    \label{alg:lsb_encode}
\end{algorithm}
\begin{algorithm}[ht!]
    \KwData{Заполненный контейнер}
    \KwResult{Сообщение в бинарном представлении}
    $Message \leftarrow$ Пустой список\;
    $Container \leftarrow$ Массив с элементами контнейнера\;
    $N \leftarrow$ Длина $Container$\;
    \For{$i = 1, 2, \ldots, N$}{
        \eIf{$Container[i] \equiv 0 \pmod{2}$}{
            $Message.append(0)$\;
        }{
            $Message.append(1)$\;
        }
    }
    \caption{LSB Декодирование}
    \label{alg:lsb_decode}
\end{algorithm}

Как видно, сначала сообщение преобразуется в бинарный вид,
а затем кодируется в элементах контейнера за счет изменения четности младшего бита.
Логические операции в данном случае соответствуют бинарным операциям на компьютере.
В итоге последние биты элементов контейнера в точности повторяют сообщение.
Так же можно заметить, что единственная часть алгоритма, зависящая от контейнера
--- это выделение массива элементов из контейнера. Алгоритм~\ref{alg:lsb_decode}
показывает, как декодировать сообщение из заполненного стегоконтейнера.

Реализуем этот алгоритм в виде класса на Python. Как уже было сказано,
существенная часть алгоритма не зависит от контейнера,
поэтому целесообразно реализовать алгоритм как абстрактный класс,
от которого будут наследоваться реализации для конкретных контейнеров.
\lstinputlisting[language=Python, style=simplecode, caption=Абстрактный класс LSB, frame=single]{Code/lsb.py}

Как видно, в методах нет циклов \textbf{for}. Они скрыты за интерфейсом библиотеки numpy.
Интерфейс библиотеки numpy позволяет нам применят операции к матрицам,
из-за чего код выглядет лаконичнее. К тому же библиотека написана на языке C,
поэтому ее код работает очень быстро.

Напишем реализацию LSB для PNG. Прежде чем реализовать метод LSB для PNG-контейнера,
имеет смысл коротко изложить формат данных PNG.

\section{Формат файла PNG}

В самом общем виде PNG файл представляет из себя сигнатуру,
за которой следует последовательность блоков,
как показано на рисунке~\ref{img:png_1}
\begin{figure}[ht!]
    \caption{Общий вид формата PNG}
    \includegraphics{PNG/1}
    \centering
    \label{img:png_1}
\end{figure}

Сигнатура PNG файла состоит из 8 байт, в hex нотации они выглядят так:
\textbf{89 50 4E 47 0D 0A 1A 0A}.

Каждый блок состоит из четырех секций: длина, тип, содержание, CRC, --- как показано на рисунке~\ref{img:png_2}:
\begin{figure}[ht!]
    \caption{Общий вид чанка}
    \includegraphics{PNG/2}
    \centering
    \label{img:png_2}
\end{figure}
В длине указывается длина блока в байтах. Тип указывается с помощью четырех ascii символов,
чувствительных к регистру. С помощью регистра декодеру передает дополнительная информация, а именно:
\begin{enumerate}
    \item Регистр первого символа сообщает, является данный блок критическим или нет. Критические
    блоки распознаются каждым декодером. Если декодер не может распознать тип такого блока,
    он аварийно завершает работу.
    \item Регистр второго символа задает публичность или приватность блока.
    Публичные блоки обычно официальные и хорошо задокументированы. Чтобы закодировать в библиотека
    какую-то специфичную информацию, его тип можно изменить на приватный.
    \item Регистр третьего символа зарезервирован на будущее. По умолчанию там стоит символ в большом регистре.
    \item Регистр четвертого символа сообщает возможность копирования данного блока редакторами.
\end{enumerate}

Список критических блоков:
\begin{enumerate}
    \item IHDR --- заголовочный блок, содержающий основную информацию об изображении.
    \item PLTE --- палитра изображения.
    \item IDAT --- содержит непосредственно изображение.
    В любом PNG файле должно быть не менее одного такого блока.
    \item IEND --- завершающих чанк. Должен находиться в самом конце файла.
\end{enumerate}

Список некритических блоков:
\begin{enumerate}
    \item bKGD --- блок, задающий фоновый цвет изображения.
    \item cHRM --- блок, используемый для задания цветового пространства CIE 1931.
    \item gAMA --- определяет гамму.
    \item hIST --- хранит гистограмму изображения либо общее содержания каждого цвета в рисунке.
    \item iTXt --- содержит текст в UTF-8
    \item pHYs --- содержит размер пикселя или отношение сторон изобарежния.
    \item sRGB --- свидетельствует об использовании sRGM схемы.
    \item tIME --- дата последнего изменения изображения.
    \item tRNS --- информация о прозрачности.
\end{enumerate}

Согласно вышесказанному, минимальный PNG файл выглядит так,
как показано на рисунке~\ref{img:png_3}

\begin{figure}[ht!]
    \caption{Минимальный PNG}
    \includegraphics{PNG/3}
    \centering
    \label{img:png_3}
\end{figure}

В следующей секции представлены данные блока.
В секции CRC записан CRC блока.

Наиболее интересными для нас являются блоки с типами IHDR и IDAT.
IHDR --- заголовочный блок, который является обязательным для PNG файла.
Он содержит следующие интересующие нас поля:
\begin{enumerate}
    \item Ширина изображения в пикселях.
    \item Высота изображения в пикселях.
    \item Битовая глубина, задающее количество бит на каждый сэмпл.
    \item Тип цвета. Возможны следующие значения:
    \begin{enumerate}
        \item Градация серого
        \item RGB
        \item Индексы из палитры
        \item Градация серого и альфа-канал
        \item RGB и альфа-канал
    \end{enumerate}
\end{enumerate}

Блок IDAT содержит сжатые данные изображения.
На данный момент поддерживается только сжатия по алгоритму deflate.

\section{Реализация LSB для контейнера PNG}
PNG изображение представимо в виде матрицы, элементами которой являются пиксели.
В случае RGB каждый пиксель представляет элементы из трех каналов, каждый из каналов
в отдельности может рассматриваться как градация серого. В случае градации серого каждый
пиксель просто представлен значением от 0 до 255.
Чтобы закодировать сообщение в эту матрицу, склеим ее строки друг с другом в одну большую строку,
равно как и склеим каналы, чтобы они образовали последовательность элементов.
Именно это делает метод \textbf{\_ to \_ elements}.
Такой метод одинаково хорошо подходит и для разных типов цвета: RGB, RGB и альфа-канала,
градации серого, градации серого и альфа-канала.
Чтобы из элементов получить двумерную RGB матрицу,
проделаем обратную операцию, что и далет метод \textbf{\_ from \_ elements}.

В функции \textbf{main} используем как сообщение книгу "Алиса в стране чудес" в оригинале.
Считаем файл с книгой как последовательность байт и закодируем в изображение с помощью LSB.
Далее выполним декодирование и сверим полученные данные.
Исходный код представлен в листинге~\ref{code:png}.
\lstinputlisting[language=Python, label={code:png}, style=simplecode, caption=реализация LSB для PNG, frame=single]{Code/png.py}
Сравнение изображение до и после заполнения контейнера методом LSB
привидено на рисунке~\ref{img:lsb}.
Как видно, два рисунка визуально неотличимы,
хотя в одном из них закодировано 150 килобайт информации.
\begin{figure}[ht!]
    \centering
    \begin{subfigure}{.5\textwidth}
      \centering
      \includegraphics[width=.9\linewidth]{PNG/Lenna.png}
      \caption{Оригинал}
      \label{img:lenna-png}
    \end{subfigure}%
    \begin{subfigure}{.5\textwidth}
      \centering
      \includegraphics[width=.9\linewidth]{PNG/LSB_Lenna.png}
      \caption{После приминения LSB}
      \label{img:lenna-lsb}
    \end{subfigure}
    \caption{Изображение до и после приминения LSB}
    \label{img:lsb}
\end{figure}
\chapter{Сокрытие в спектральной области}
\section{Дискретное косинусное преобразование}
Дискретное косинусное преобразование (ДКП) --- одно из дискретных преобразований Фурье.
ДКП представляет конечную последовательность в виде суммы функций косинуса,
колеблющихся на разных частотах. ДКП широко используется при обработке сигналов и сжатии данных.
Например, ДКП используется при сжатии в изображениях (JPEG, HEIF), аудиофайлах (Dolby Digital, MP3),
видеофайлах (MPEG, H.26x), в цифровом телевидении (SDTV, HDTV, VOD) и в других.

ДКП является линейным ортогональным преобразованием. Как и любое дискретное линейное преобразование,
ДКП можно представить в виде матрицы. Будучи ортоганальным преобразованием, обратное к ДКП преобразование
задается транспонированной матрицей ДКП, домноженной на какой-то коэффициент.

Использование косинусных, а не синусоидальных функций имеет решающее значение для сжатия,
поскольку для аппроксимации типичного сигнала требуется меньше косинусных функций.
ДКП подобно дискретному преобразованию Фурье, но использует только действительные числа.

Существует 8 стандартных типов ДКП, однако наиболее употребимым является второй тип,
который часто называют просто ДКП (DCT-II).
Формула дискретного косинусного преобразования выглядит так,
как показано в формуле~\ref{eq:simple-dcp}:
\begin{equation} \label{eq:simple-dcp}
    X_k = \sum_{n=0}^{N-1} x_n \cos \left[\frac{\pi}{N} \left(n+\frac{1}{2}\right) k \right] \quad \quad k = 0, \dots, N-1    
\end{equation}
Формула для матрицы преобразования выглядит как в~\ref{eq:matrix-dcp}:
\begin{equation} \label{eq:matrix-dcp}
    {DCT}\text{-}2_n= \left[\cos (k(l+\tfrac{1}{2})\tfrac{\pi}{n})\right]_{0\leq k,l<n}    
\end{equation}

Как и в случае быстрого преобразования Фурье, существуют алгоритмы быстрого ДКП преобразования.

DCT-II часто используется для сжатия с потерями благодаря своему свойству уплотнения энерегии:
в типичных случаях большая часть информации, которую содержит сигнал, концентрируется в нескольких
первых коэффициентах разложения.

Существуют так же многомерные ДКП, которые получаются из одномерных путем композиции ДКП по каждому измерению.
Вывод такого преобразования для двумерного случая показан в формуле~\ref{eq:2d-dcp}.
\begin{align}
    X_{k_1,k_2} &= \nonumber
    \sum_{n_1=0}^{N_1-1}
    \left( \sum_{n_2=0}^{N_2-1}
    x_{n_1,n_2} 
    \cos \left[\frac{\pi}{N_2} \left(n_2+\frac{1}{2}\right) k_2 \right]\right)
    \cos \left[\frac{\pi}{N_1} \left(n_1+\frac{1}{2}\right) k_1 \right]\\
    &= \sum_{n_1=0}^{N_1-1}
    \sum_{n_2=0}^{N_2-1}
    x_{n_1,n_2} 
    \cos \left[\frac{\pi}{N_1} \left(n_1+\frac{1}{2}\right) k_1 \right]
    \cos \left[\frac{\pi}{N_2} \left(n_2+\frac{1}{2}\right) k_2 \right] \label{eq:2d-dcp}
\end{align}
Здесь $[x_{n_1,n_2}]$ --- матрица до преобразования, и $[X_{k_1,k_2}]$ --- матрица
после преобразования.
В матричном виде это преобразование может быть представлено так, как показано
в формуле~\ref{eq:2d-matrix-dcp}, где $x$ --- матрица, которую нужно преобразовать.
\begin{equation} \label{eq:2d-matrix-dcp}
    X = ({DCT}\text{-}2_n) x ({DCT}\text{-}2_n ^ T)
\end{equation}
Именно такое преобразование используется при компрессии в JPEG.

\section{JPEG}
JPEG является широко используемым методом сжатия с потерями для цифровых изображений.
Степень сжатия регулируется, что позволяет выбирать между качеством и размером изображения.
JPEG также является наиболее широко используемым стандартом сжатия изображений в мире и
наиболее используемым форматом цифровых изображений.

ДКП лежит в основе сжатия методом JPEG. Как уже говорилось выше,
ДКП был выбран именно благодаря свойству уплотнения энергии. Чтобы прояснить,
о чем идет речь, мной была создана визуализация преобразования ДКП.

Выберем на изображении область 32x32 пикселя, как показано на рисунке~\ref{img:lenna-eye}.

\begin{figure}[ht!]
    \centering
    \includegraphics[width=\linewidth]{DCT/Lenna_eye.png}
    \caption{Выбираем область}
    \label{img:lenna-eye}
\end{figure}

Для простоты будем использовать только синий канал изображения.
Сначала рассмотрим матрицу пикселей как двумерную дискретную функцию.
Расположим координаты так, чтобы в левом верхнем углу
располагался пиксель с координатами $p_{kj}, k = 0, j = 0$.
Визуализацию можно посмотреть на рисунке~\ref{img:pixels-dct}.

\begin{figure}[ht!]
    \centering
    \caption{Визуализация пикселей}
    \includegraphics[width=\linewidth]{DCT/pixels.png}
    \label{img:pixels-dct}
\end{figure}

Умножим эту функцию справа на транспонированную матрицу ДКП
по формуле~\ref{eq:2d-matrix-dcp}. Мы получим новую функцию,
которая показана на рисунке~\ref{img:dct-1}.
Таким образом, фактически ДКП применилось к каждой строке матрицы.
Из изображения видно, что наибольшие коэффициенты расположены
в нижней части спектра, то есть ближе к нулевому столбцу.

\begin{figure}[ht!]
    \centering
    \caption{После приминения ДКП к строкам матрицы}
    \includegraphics[width=\linewidth]{DCT/dct-1.png}
    \label{img:dct-1}
\end{figure}

К полученной матрице применим ДКП еще раз, в этот раз по столбцам.
В полученной матрице наибольшее значение имеет коэффициент с координатами
$k = 0, j = 0$. Этот коэффициент называется DC-коэффициент.
Остальные коэффициенты называются AC-коэффициентами.
Матрица показана на рисунке~\ref{img:dct-2}.

\begin{figure}[ht!]
    \centering
    \caption{После приминения ДКП к столбцам матрицы}
    \includegraphics[width=\linewidth]{DCT/dct-2.png}
    \label{img:dct-2}
\end{figure}

DC-коэффициент блока равен среднему всех пикселей в блоке,
взятому с определенным коэффициентом. Удаляя все коэффициенты,
кроме DC, мы можем аппроксимировать блок пикселей их средним
арифметическим. Чем дальше коэффициент располагается от DC,
тем меньше психовизуальной информации он несет для человека,
и тем более незаметные детали изображения он хранит в себе.
Соответственно, основная идея алгоритма состоит в отбрасывании
наименее значимых коэффициентов. Это позволяет производить сжатие
изображения с потерями.

Алгоритм сжатия JPEG работает с каждым каналом отдельно, поэтому для
простоты рассмотрим работу JPEG на изображении в режиме градации серого.
В самом начале своей работы алгоритм разбивает изображение на блоки
8x8 пикселей. К каждому блоку применяется ДКП преобразование,
что равносильно разложению исходной матрицы по базису, состоящему
из 64 функций. Эти 64 функции показаны на рисунке~\ref{img:basis}.
Из этого рисунка видно, что по мере отдаления от левого верхнего угла
функции становятся все более рельефными, что объясняет, почему они несут
наиболее мелкие детали изображения. Также видно, что функция,
соответствующая DC-коэффициенту, представлена плоскостью. Очевидно,
что лучшим константным приближением функции
является ее математическое ожидание.
\begin{figure}[ht]
    \centering
    \caption{Базис ДКП}
    \includegraphics[width=.9\linewidth]{DCT/basis.png}
    \label{img:basis}
\end{figure}
После применения ДКП преобразования к блоку матрицы 8x8
получается другая матрица той же размерности. В соответствии с вышесказанным,
эта матрица делится на области низких, средних и высоких частот. В таком порядке
убывает информативность коэффициентов. Это можно увидеть на рисунке~\ref{img:freq}.
\begin{figure}[ht]
    \centering
    \caption{Частотные области ДКП}
    \includegraphics[width=.8\linewidth]{DCT/freq.png}
    \label{img:freq}
\end{figure}
Далее коэффициенты полученной ДКП матрицы квантуются.
Квантование происходит с приминением специальных матриц,
одна из таких матриц представлена в формуле~\ref{eq:q-matrix}.
\begin{equation} \label{eq:q-matrix}
Q=
\begin{bmatrix}
 16 & 11 & 10 & 16 & 24 & 40 & 51 & 61 \\
 12 & 12 & 14 & 19 & 26 & 58 & 60 & 55 \\
 14 & 13 & 16 & 24 & 40 & 57 & 69 & 56 \\
 14 & 17 & 22 & 29 & 51 & 87 & 80 & 62 \\
 18 & 22 & 37 & 56 & 68 & 109 & 103 & 77 \\
 24 & 35 & 55 & 64 & 81 & 104 & 113 & 92 \\
 49 & 64 & 78 & 87 & 103 & 121 & 120 & 101 \\
 72 & 92 & 95 & 98 & 112 & 100 & 103 & 99
\end{bmatrix}.
\end{equation}
Это матрица для 50\% качества, указанная в исходном стандарте JPEG.
Каждый элемент ДКП матрицы делится на элемент матрицы квантования,
стоящий в той же позиции. После этого результат округляется. В результате
этой операции обычно бывает так, что многие высокочастотные компоненты
округляются до нуля, а многие из остальных становятся небольшими положительными
или отрицательными числами, для представления которых требуется гораздно меньше бит.

При декодировании происходит обратный процесс --- матрица квантованных коэффициентов
почленно умножается на матрицу квантования, но из-за того, что до этого значения были округлены,
они восстанавливаются с некоторой погрешностью. Чем больше коэффициент квантования,
тем больше будет эта погрешность.

После квантования коэффициенты записываются в специальном зигзагообразном порядке,
показанном на рисунке~\ref{img:zigzag}. Таким образом коэффициенты упорядочиваются
от низких частот к высоким.
\begin{figure}[ht]
    \centering
    \caption{Зигзагообразный порядок}
    \includegraphics[width=.5\linewidth]{DCT/zigzag.png}
    \label{img:zigzag}
\end{figure}
После этого DC и AC коэффициенты кодируются отдельно.
Поскольку в изображениях часто встречаются
градиентные области, то DC коэффициенты соседних блоков
скоррелированны, поэтому первым этапом их кодирования
становится дифференциальная импульсно-кодовая модуляция (ДИКМ).
То есть кодируются не сами коэффициенты, а разница между двумя
соседними коэффициентами.
AC коэффициенты кодируются с помощью кодирования длин серий (КДС).
То есть повторяющиеся символы заменяются на символ и количество его повторов.
После этого к DC и AC коэффициентам применяется энтропийное кодирование с помощью
алгоритма Хаффмана.

Мы разобрали работу алгоритма для одноканального изображения.
В RGB изображениях такой алгоритм применяется к каждому каналу отдельно.
Также часто кодированию предшествует дополнительный этап,
на котором RGB преобразуется в цветовое пространство
YC\textsubscript{B}C\textsubscript{R}.
Дело в том, что человеческий глаз более чувствителен к перепадам
яркости, чем к перепаду цвета.
В YC\textsubscript{B}C\textsubscript{R} первый канал Y отвечает за яркость,
C\textsubscript{B} и C\textsubscript{R} отвечают за синий и красный компоненты.
После преобразования в YC\textsubscript{B}C\textsubscript{R}
над C\textsubscript{B} и C\textsubscript{R} производится субдискретизация:
каналы разбиваются на небольшие блоки и значения пикселей в этих блоках усредняются.
Таким образом разрешение в этих каналах понижается еще сильнее
и изображение сжимается еще сильнее.
Вся схема алгоритма представлена на рисунке~\ref{img:jpeg-alg}
\begin{figure}[ht]
    \centering
    \caption{Схематичное изображение JPEG}
    \includegraphics[width=.8\linewidth]{DCT/jpeg-alg.png}
    \label{img:jpeg-alg}
\end{figure}

\section{JSteg}
JSteg --- стеганографический алгоритм, работающий с JPEG файлами.
JSteg во многом опирается на работу кодировщика JPEG. Алгоритмы
кодирования и декодирования JPEG абсолютно симметричны.
JSteg вмешивается в в работу декодировщика, а именно прерывает
его на этапе умножения ДКП коэффициентов на матрицу квантования.
После этого JSteg записывает в упорядоченные зигзагообразным образом
ДКП коэффициенты кодируемую информацию методом LSB. После этого вызывается
кодировщик, который записывает измененные ДКП коэффициенты обратно в JPEG
изображение.

Рассмотрим достоинства и недостатки этого метода.
К доистоинствам можно отнести следующее:
\begin{enumerate}
    \item Низкая вычислительная сложность.
    \item Алгоритм обеспечивает большую вместимость стегосообщений:
    стегосообщение может занимать до 13\% объема контейнера.
    \item Изменения, вносимые в контейнер, незаметны для человеческого глаза.
\end{enumerate}
Но у метода также есть и существенные недостатки:
\begin{enumerate}
    \item Метод неустойчив к квантованию ДКП коэффициентов.
    Как уже говорилось ранее, операция квантования
    восстанавливает и сохраняет коэффициент с некоторой погрешностью,
    поэтому если открыть в редакторе стегоконтейнер, в котором
    содержится сообщение, закодированное методом JSteg,
    то после пересохранения этого файла сообщение полностью уничтожится. 
    \item Из предыдущих соображений становится ясно, что метод неустойчив
    к сжатию. Это также обусловлено еще и тем, что при сжатии коэффициенты
    квантования увеличиваются, а значит часть ДКП коэффициентов обнулится,
    а у другой части погрешность восстановления станет еще больше.
\end{enumerate}
Рассмотрим реализацию метода на Python. Код приведен в листинге~\ref{code:jsteg}.
По сути метод представляет собой LSB, только вместо пространственной области
используется спектральная. В данном случае наименее значимый бит меняется
у коэффициентов ДКП изображения, упорядоченных зигзагообразным способом. 
\lstinputlisting[language=Python, label={code:jsteg}, style=simplecode, caption=Реализация JSteg, frame=single]{Code/jsteg.py}

\section{Метод относительной замены величин коэффициентов ДКП}
Этот метод использует идеи, схожие с методами расширения спектра, а именно:
вместо того, чтобы кодировать 1 бит информации в одном коэффициенте ДКП,
метод предлагает кодировать 1 бит информации за счет нескольких коэффициентов ДКП.
Этого можно добиться, кодируя информацию за счет изменения разности между
набором различных коэффициентов ДКП.

Алгоритм Коха-Жао использует 2 коэффициента ДКП.
Формальное описание приводится в алгоритме~\ref{alg:koch-jao}.
Алгоритм декодирования строится симметрично.
Это простейший метод из данного семейства. К доистоинствам
метода можно отнести то, что он устойчив к квантованию ДКП-коэффициентов
и сжатию. Особенно если применять его в паре с помехоустойчивым кодированием.
Но у метода также есть и серьезные недостатки:
\begin{enumerate}
    \item Метод вносит заметные искажения в контейнер.
    \item Метод легко детектируется.
\end{enumerate}

\begin{algorithm}[ht!]
    \KwData{Контейнер, Сообщение}
    \KwResult{Заполненный стегоконтейнер}
     $N \leftarrow$ Длина сообщения в битах\;
     $Message \leftarrow$ Бинарное представление сообщения\;
     $DCT$-$blocks \leftarrow$ Массив из блоков ДКП контейнера\;
     $k, l \leftarrow$ Позиция коэффициента ДКП из низкой полосы частот\;
     \For{$i = 1, 2, \ldots, N$}{
         \eIf{$Message[i] = 0$}{
             Сделать $|DCT$-$blocks[i][k][l] - DCT$-$blocks[i][k][l]| < 25$\;
         }{
             Сделать $|DCT$-$blocks[i][k][l] - DCT$-$blocks[i][k][l]| > 25$\;
         }
     }
     $Container \leftarrow$ Новый контейнер, полученный из обратного преобразования ДКП-блоков\;
     \caption{Алгоритм Коха-Жао}
    \label{alg:koch-jao}
\end{algorithm}

Модифицированной версией этого метода является метод Бенгама-Мемона-Эо-Юнга.
Модификации подверглись два направления:
\begin{enumerate}
    \item Встраивание происходит только в наиболее подходящие ДКП-блоки.
    \item Используются не 2, а 3 коэффициента ДКП. Это существенно снижает
    вносимые в контейнер искажения.
\end{enumerate}.
Рассмотрим каждую модификацию в отдельности.

Наиболее подходящие коэффициенты выбираются по следующим признакам:
\begin{enumerate}
    \item Блок не должен иметь слишком резких переходов яркости.
    \item Блок не должен быть слишком монотонным.
\end{enumerate}
Для оценки этих параметров вводится два коэффициента: P\textsubscript{L} и P\textsubscript{H}.
Превышение первого коэффициента или недостижение второго будет указывать на то,
что блок не пригоден для встраивания. Для получения первой оценки нужно проссумировать модуляции
низкочастотных коэффициентов, а для получения второй оценки нужно проссумировать модули высокочастотных
коэффициентов.

Само встраивание происходит в два этапа.
На первом этапе выбираются три коэффициента из низкой полосы частот.
Для обеспечения большой стойкости они могут выбираться псевдослучайно.
На втором этапе коэффициенты модифицируются. Если кодируется 0,
то третий коэффициент должен стать больше первых двух, а если 1,
то третий коэффициент должен стать меньше, соответственно.
После встраивания запоминаются номера блоков, в которых закодирована информация.
Декодирование происходит симметричным образом с той лишь разницей,
что вместо поиска подходящих блоков используются запомненные на прыдедущем этапе блоки.
Реализацию метода на Python можно найти в приложении~\ref{appendix-a}.
\chapter{Стегоанализ}
\section{Методы стегоанализа}
Осноными методами, применяемыми в стегоанализе, являются визуальные методы и статистические методы.

Субъективная атака проста по своей сути: аналитик пытается "на глаз" определить, содержит ли контейнер стего.
Однако эта атака может применяться в различных вариациях. Например, анализу может подвергаться не само изображение,
а какой-то его канал, или же изображение, полученное из данного отбрасыванием нескольких старших бит.

Статистические методы основаны на использовании различных статистик изображения и том факте,
что эти статистики могут различаться для пустого и заполненного стегоконтейнера.

\section{Субъективная атака LSB}
Метод LSB является легко обнаружимым.
Для начала рассмотрим самую простую ситуацию: допустим,
что сообщение, скрытое в стего, не зашифровано.
В таком случае стегоконтейнер поддается визуальной атаке,
а именно: попробуем посмотреть на визуаьное представление
последнего бита сообщения. Для этого используем код,
представленный в листинге~\ref{code:visual-lsb}.
\lstinputlisting[language=Python, label={code:visual-lsb}, style=simplecode, caption=Визуализация LSB, frame=single]{Code/visual_lsb.py}
Полученные изображения можно увидеть на рисунке~\ref{img:bw-lsb}.
Как видно, наименее значащий бит оригинального изображения распределен
шумоподобно, в то время как у стегоконтейнера этот бит
вносит в изображение какую-то струкуру. Так происходит из-за
неравномерного распределения символов в исходном сообщении.
Так же из изображения можно увидеть примерную длину сообщения.
\begin{figure}[ht!]
    \centering
    \begin{subfigure}{.5\textwidth}
      \centering
      \includegraphics[width=.9\linewidth]{PNG/BW_Lenna.png}
      \caption{Оригинал}
      \label{img:bw-lenna-png}
    \end{subfigure}%
    \begin{subfigure}{.5\textwidth}
      \centering
      \includegraphics[width=.9\linewidth]{PNG/BW_LSB_Lenna.png}
      \caption{После приминения LSB}
      \label{img:bw-lenna-lsb}
    \end{subfigure}
    \caption{Наименее значащий бит до и после LSB}
    \label{img:bw-lsb}
\end{figure}

\section{Атака оценки числа переходов значений младших бит в соседних элементах контейнера}
Допустим, что стегосообщение предварительно было зашифровано и сокрыто методом LSB.
В таком случае сообщение внутри контейнера будет обладать предельной энтропией,
а это значит, что наименее значимый бит контейнера будет распределен равномерно:
50\% объема будет занимать 0 и другие 50\% будет занимать 1. При этом два соседних
наименее значимых бита не будут скорелированны. Однако в реальных изображениях это не так.
В реальном изображении соседние пиксели скореллированы, и ненулевая вероятность того,
что они окажутся одинаковыми. Несмотря на то, что изображение~\ref{img:bw-lenna-png}
выглядит как шум, в нем есть некоторая статистическая структура. Чтобы выявить ее,
нужно попарно сравнить 2 соседних пикселя и посчитать, какой процент этих пикселей совпал.
Такое сравнение проводит в листинге~\ref{code:lsb-correlation}.
\lstinputlisting[language=Python, label={code:lsb-correlation}, style=simplecode, caption=Корреляция LSB, frame=single]{Code/lsb_correlation.py}
Вывод программы следующий: "Original: P(False) = 0.46, P(True) = 0.54; Stego: P(False) = 0.5, P(True) = 0.5".
Как видно, у оригинального сообщения совпадение и несовпадение двух соседних наименее значимых битов не равновероятны. 

\section{Атака хи-квадрат}
Допустим, что мы оказались в похожей ситуации: зашифрованное сообщение
было сокрыто методом LSB, --- но в этот раз в контейнере соседние элементы не скорелированны.
Построим гистаграмму элементов контейнера и рассмотрим элементы, отличающиеся друг от друга в младшем бите.
Если это незаполненный стегоконтейнер, то частота двух соседних элементов $a$ и $b$ может сильно отличаться.
Однако если к такому контейнеру применить LSB, то в паре двух соседних значений $a$ в 50\%
случаев не изменится, и в 50\% случаев изменится на единицу. То же самое произойдет и с $b$.
Допустим, что $a$ был четным. Тогда в новом распределении частота $a$ будет равна $\dfrac{ f(a) + f(b)}{2}$,
где $f$ -- функция распределения частот в незаполненном контейнере. То же самое касается и $b$.
То есть соседние элементы будут распределены одинаково. На этом и основана атака хи-квадрат.

Критерий согласия Пирсона (критерий согласия $\chi^2$) --- это критерий принадлежности наблюдаемой
выборки $x_1, x_2, \dots , x_3$ теоретическому закону распределения $F(x, \theta)$,
где $\theta$ --- это известный параметр распределения.
Процедура проверки гипотез с использованием критерия$\chi^2$ предусматривает группирование наблюдений.
Область определения случайной величины разбивают на $k$
непересекающихся интервалов граничными точками
$x_{(0)}, x_{(1)}, \dots, x_{(k)}$.
В соответствии с заданным разбиением подсчитывают число $n_i$ выборочных значений,
попавших в i-й интервал, и вероятности попадания в интервал
$P_i(\theta) = F(x_{(i)}, \theta) - F(x_{(i - 1)}, \theta))$
соответствующие теоретическому закону с функцией распределения $F(x, \theta)$.
При этом $n = \sum_{i = 1}^{k} n_i$ и $\sum_{i = 1}^{k} P_i(\theta) = 1$.
В основе критерия согласия Пирсона лежит измерение отклонений
$\dfrac{n_i}{n}$ от $P_i(\theta)$.
Статистика критерия согласия $\chi^2$ Пирсона определяется соотношением~\ref{eq:chi-2}.
\begin{equation}
    \label{eq:chi-2}
    \chi^2 = n \sum_{i=1}^k \dfrac{(n_i / n - P_i(\theta))^2}{P_i(\theta)}
\end{equation}

Выделим из изображения все пары элементов, которые отличаются только в младшем бите.
Обозначим такие пары через $(2m, 2m + 1)$. Пусть $h_i$ обозначает гистограмму наблюдаемого
распределения элементов.
Введем новое наблюдаемое распределение $\{o_m\}$ равное $o_m = h_{2m}$
и теоретическое распределение $\{e_m\}$ равное $e_m = \dfrac{h_{2m} + h_{2m + 1}}{2}$.
Разница между этими двумя распределениями измеряется критерием~\ref{eq:chi-stego}
с $(\nu - 1)$ степенями свободы, где $\nu$ --- количество пар, отличающихся в младшем бите.
\begin{equation}
    \label{eq:chi-stego}
    \chi^2 = \sum_{e_m \neq 0} \dfrac{(o_m - e_m) ^ 2}{e_m}
\end{equation}

Степень сходства двух распределений $\{o_m\}$ и $\{e_m\}$ после этого считается
с помощью функции распределения по формуле~\ref{eq:chi-cdf}.
\begin{equation}
    \label{eq:chi-cdf}
    p = 1 - \int_{0}^{\chi^2} \dfrac{t^{(\nu - 2)/2}e^{-t/2}}{2^\nu\Gamma(\nu/2)}dt
\end{equation}

Реализация атаки на python представлена в листинге~\ref{code:chi}.
\lstinputlisting[language=Python, label={code:chi}, style=simplecode, caption=Атака хи-квадрат, frame=single]{Code/chi.py}
Программа выводит 0.0 для пустого контейнера и 0.99 для заполненного.

\section{Методы противодействия}
Для эффективного противодействия описанным статистическим и субъективным
атакам рекомендуется пользоваться следующими рекомендациями при построении
стеганографического алгоритма:
\begin{enumerate}
    \item Сообщение должно встраиваться в зашифрованном виде для предотвращения
    субъективных и иных атак.
    \item Сообщение должно встраиваться не в подряд идущие элементы контейнера,
    а в случайно выбарнные элементы, например, с использованием генератора псевдослучайных
    чисел.
    \item Сообщение должно заполнять лишь малую часть емкости контейнера. Иначе
    оно может нарушить статистическую структуру контейнера. Так, например,
    атака хи-квадрат работает намного хуже, когда сообщение рассеяно по всей длине контейнера.
\end{enumerate}
\chapter{Экономическая оценка проекта}
\section{Постановка задачи}
Целью дипломного проекта является
анализ стеганографических методов защиты информации.
Данный раздел содержит расчет трудоемкость и затрат на
проведение анализа предметной области, разработки и сопровождения
стеганографической системы.
\section{Оценка стоимости объектов интеллектуальной собственности}
Оценка стоимости объектов интеллектуальной собственности (ОИС),
созданных на предприятии или по его заказу (при финансировании разработок
предприятием) с закреплением за ним по договору прав собственности на них,
производится по затратному методу и определяется по формуле~\ref{eq:eco-1}:
\begin{equation}
    \label{eq:eco-1}
    C_{i} = C_\textup{р} + C_\textup{п} + C_\textup{м}
\end{equation}
где,
\begin{enumerate}
    \item $C_\textup{р}$ --- приведенные затраты на создание объектов интеллектуальной собственности, руб.;
    \item $C_\textup{п}$ --- привиденные затраты на правовую охрану объектов интеллектуальной собственности, руб.;
    \item $C_\textup{м}$ --- привиденные затраты на маркетинговые исследования ОИС, руб.
\end{enumerate}

Приведенные затраты на создание ОИС --- сумма фактически произведенных затрат
на выполнение научно-исследовательской работы (НИР) в полном объеме
(от поиска материалов исследования до формирования отчета)
и разработку всей технической документации.
Приведенные затраты для НИР состоят из затрат на поисковые работы,
включая предварительную проработку проблемы, на теоретические исследования,
на проведение экспериментов, испытаний, на услуги сторонних организаций,
на составление, рассмотрение и утверждение отчета и прочих затрат.

Приведенные затраты на разработку технической документации (ТД) состоят из затрат
на выполнение эскизного проекта, технического задания, рабочего проекта,
расчетов, испытаний, услуг сторонних организаций, авторского надзора, дизайна.
Кроме того, сюда включаются затраты на доведение ОИС до готовности
промышленного использования и коммерческой реализации.

В тех случаях, когда НИР или технологическая и проектная документация выполняется частично
или созданию ОИС предшествует проведение только НИР или разработка технической документации,
то расчет стоимости ОИС производится по затратам на фактически выполненные работы,
для товарных знаков и промышленных образцов --- затратам на дизайн.

Приведенные затраты на правовую охрану ОИС --- затраты на оформление
заявочных материалов на получение патента (свидетельства), переписка по заявке,
оплата пошлин за проведение экспертизы, получение патента (свидетельства)
и поддержание его в силе и т.д. Данная составляющая отсутствует для таких ОИС, как ноу-хау, НИР, ТД.

Приведенные затраты на маркетинговые исследования --- для целей приведения
разновременных стоимостных оценок к конечному году применяется коэффициент $\alpha_i$,
но данном случае он будет равен 1, потому что число лет, предшествующих расчетному году равно 0.
Затраты, произведенные на выполнение НИР и разработку всех стадий ТД, то есть Cp могут включать в себя:
\begin{enumerate}
    \item израсходованные материальные ресурсы;
    \item оплата труда с отчислениями разработчиков;
    \item амортизационные отчисления оборудования, которое использовалось при разработке ОИС;
    \item аренда помещения для разработчиков;
\end{enumerate}
Кроме того, создание любого ОИС или разработка программного обеспечения происходит
не всегда согласно производственного задания, где четко оговорены сроки работы.
Разработчик может вне плановых заданий выполнить(изготовить) ОИС или написать программу для ЭВМ.
В случае если отсутствует документация по затратам на создание ОИС, то их можно определить расчетным путем.

Затраты на создание ОИС в t-том году определяются по следующей формуле:
\begin{equation}
    \label{eq:eco-2}
    C_\textup{пр} = \dfrac{\textup{З}}{m}Kt
\end{equation}
Где,
\begin{enumerate}
    \item З --- среднемесячная заработная плата разработчика (разработчиков) с учетом районного коэффициента, руб.;
    \item m --- среднее количество рабочих часов в месяце;
    \item K --- коэффициент, учитывающий отчисления с заработной платы (страховые взносы).
    Ставка страховых взносов в 2021 г. составляет 30\% от величины фонда оплаты труда. K = 0,3; 
    \item t --- время в часах, затрачиваемое разработчиком (разработчиками) на создание (разработку) ОИС,
    на отладку и адаптацию ОИС к условиям производства.
\end{enumerate}

Для расчета себестоимости необходимы затраты времени.
Весь перечень произведённых работ не был оговорен рамками технического задания
с указанием конкретных сроков выполнения. Для обеспечения наибольшей достоверности
временных затрат используем метод экспертных оценок. Время работы, согласно плану
проведения аудита ИБ учтём в соответствующих этапах.

Для определения средних значений $a_\textup{i ср}$, $m_\textup{i ср}$, $b_\textup{i ср}$
используются экспертные оценки, данные руководителем и автором проекта.
Средние значения найдем по формуле\ref{eq:eco-4}.
Значения $m_i$ и $b_i$ рассчитываются аналогично.
\begin{equation}
    \label{eq:eco-3}
    a_\textup{i ср} = \dfrac{3a_\textup{i рук} + 2a_\textup{i авт}}{5}
\end{equation}
где,
\begin{enumerate}
    \item $a_\textup{i рук}$ --- оценка, данная руководителем;
    \item $a_\textup{i авт}$ --- оценка, данная автором.
\end{enumerate}

\begin{xltabular}{\textwidth}{|p{0.30\textwidth}|X|X|X|X|X|X|X|X|X|}
    \caption{Затраты времени на разработку ОИС}
    \label{tab:tabular03}
    \\ \hline
    \multirow{3}{*}{Этапы} & \multicolumn{9}{c|}{Величина затрат} \\ \cline{2-10} 
     & \multicolumn{3}{l|}{Минимальная} & \multicolumn{3}{l|}{Вероятная} & \multicolumn{3}{l|}{Максимальная} \\ \cline{2-10} 
     & \rotatebox[origin=c]{90}{ Руководитель } & \rotatebox[origin=c]{90}{ Автор } & \rotatebox[origin=c]{90}{ Средняя } & \rotatebox[origin=c]{90}{ Руководитель } & \rotatebox[origin=c]{90}{ Автор } & \rotatebox[origin=c]{90}{ Средняя } & \rotatebox[origin=c]{90}{ Руководитель } & \rotatebox[origin=c]{90}{ Автор } & \rotatebox[origin=c]{90}{ Средняя } \\ \hline
    Ознакомление с исходными данными & 3 & 4 & 3.5 & 4 & 6 & 5 & 5 & 7 & 6 \\ \hline
    Анализ предметной области & 9 & 17 & 13 & 13 & 25 & 18 & 25 & 41 & 33 \\ \hline
    Разработка системы & 179 & 210 & 190 & 250 & 310 & 280 & 330 & 410 & 370 \\ \hline
    Вывод в эксплуатацию & 9 & 17 & 13 & 17 & 31 & 24 & 25 & 41 & 33 \\ \hline
\end{xltabular}

Ожидаемая величина затрат для i-го этапа ($MO_i$) и стандартное отклонение этой величины
каждого i-го этапа ($G_i$):
\begin{equation}
    \label{eq:eco-4}
    MO_i = \dfrac{\alpha_i + 4m_i + b_i}{6}
\end{equation}
\begin{equation}
    \label{eq:eco-5}
    G_i = \dfrac{b_i - a_i}{6}
\end{equation}
Результаты показаны в таблице~\ref{tbl:eco-2}

\newcolumntype{L}[1]{>{\raggedright\arraybackslash}p{#1}}

\begin{xltabular}{\textwidth}{|L{0.3\linewidth}|L{0.1\linewidth}|L{0.1\linewidth}|L{0.1\linewidth}|X|X|}
    \caption{Затраты времени на разработку ОИС}
    \label{tbl:eco-2}
    \\
    \hline
        \multirow{2}{*}{\parbox{1.0\linewidth}{Этапы разработки ОИС}} &
        \multicolumn{3}{L{0.3\linewidth}|}{Средняя величина затрат времени этапа разработки ОИС} &
        \multirow{2}{*}{$MO_i$} &
        \multirow{2}{*}{$G_i$} \\
    \cline{2-4}
        &
        $a_i$ &
        $m_i$ &
        $b_i$ &
        & \\
    \hline
        Ознакомление с исходными данными &
        3.5 &
        5 &
        6 &
        4.9 &
        0.4 \\
    \hline
        Анализ предметной области &
        13 &
        18 &
        33 &
        19.6 &
        3.3 \\
    \hline
        Разработка системы &
        190 &
        280 &
        370 &
        280 &
        30 \\
    \hline
        Вывод системы в опытную эксплуатацию &
        13 &
        24 &
        33 &
        23.6 &
        3.3 \\
    \hline
\end{xltabular}
Зная ожидаемые затраты и стандартное отклонение по каждому этапу,
рассчитываются эти показатели в целом по ОИС:
\begin{equation}
    MO = \sum_{i=1}^{n}MO_i
\end{equation}
\begin{equation}
    G = \sqrt{\sum_{i=1}^{n}G_i^2}
\end{equation}

\begin{align*}
    MO = 4.9 + 19.6 + 280 + 23.6 = 328.1
\end{align*}
\begin{align*}
    G = \sqrt{0.4^2 + 3.3^2 + 30^2 + 3.3^2} = 30.4
\end{align*}
\section{Оценка стоимости разработки}
Необходимо определить себестоимость разработки системы.
Стоимость разработки ОИС найдём по формуле~\ref{eq:eco-2} при следующих данных:
\begin{enumerate}
    \item среднемесячная заработная плата разработчика с учетом коэффициента составляет 90 000 руб.;
    \item среднее количество рабочих часов в месяце --- 168;
    \item затраты времени на разработку ОИС: $328.1 + 30 = 358.1$ часа.
\end{enumerate}

\begin{align*}
    \textup{С} = \dfrac{90000}{168}*358.1 = 191839.3
\end{align*}

Прочие расходы:
\begin{enumerate}
    \item Ставка страховых взносов в 2021 г. составляет 30\% от величины фонда оплаты труда.
    В нашем примере они составят:
    Страховые взносы = $9000*0.3 = 27000$
\end{enumerate}

Общие затраты на разработку составят:
\begin{align*}
    \textup{З} = \textup{З}_\textup{прям} + \textup{З}_\textup{пр} = 191839.3 + 27000 = 218839.3
\end{align*}

\section{Оценка стоимости использования оборудования и сопровождения системы}
Для развертывания системы будет использоваться выделенный сервер.
Траты для сервера с конфигурацией: четырехядерный процессор с восмью потоками, 16 гигабайт оперативной
памяти, 1 терабайт SSD --- 6 000 рублей/месяц. Сопровождением системы займется системный администратор.
По формуле выше найдем затраты на сотрудника при учете месячного оклада 40 000 рублей:
\begin{align*}
    \textup{С} = 40000 + 40000 * 0.3 =  52000 \textup{руб.}
\end{align*}
Суммарные среднемесячные затраты на систему составляют 58000 рублей.

\section{Экономическое обоснование проекта}
Стоимость разработки проекта составляет 218 839.3 рублей.
Анализ показал, что данная система позволит сократить убытки
компании от несанкционированного копирования и распространения цифровых объектов
в 2 раза. В среднем компания теряет 240 000 рублей в месяц из-за несанкционированного
копирования и распространения цифровых объектов компании. Рассчитаем, сколько проект приносит прибыли в месяц:
\begin{align*}
    \textup{P} = \dfrac{240000}{2} - 52000 = 68000
\end{align*}
где P --- это прибыль.
Итого срок окупаемости проекта составляет три месяца.

\backmatter %% Здесь заканчивается нумерованная часть документа и начинаются ссылки
\Conclusion
После проделанной работы можно подвести следующие итоги:
\begin{enumerate}
    \item были описаны общие сведения о стеганографических методах сокрытия информации;
    \item был описан алгоритм сокрытия информации методом наименее значимого бита;
    \item была написана реализация алгоритм сокрытия информации методом наименее;
    значимого бита с использования языка программирования Python;
    \item был проведен анализ разобранного алгоритма, найдены и продемонстрированы
    его уязвимости;
    \item были описаны и проанализированы алгоритмы сокрытия в спектральной области;
    \item был проведен анализ алгоритмов сокрытия в спектральной области,
    найдены и продемонстрированы их уязвимости;
\end{enumerate}

Из вышесказанного можно сделать вывод о том, что мной успешно были
рассмотрены основные стеганографические алгоритмы, определены области их приминения,
разобраны их недостатки и достоинства. Я научился проводить анализ стеганографических алгоритмов.
\bibliographystyle{ugost2008}
\bibliography{rpz.bib}
\nocite{*}

\appendix   % Тут идут приложения
\chapter{Реализация метода  Бенгмана-­Мемона-­Эо-­Юнга на Python}
\label{appendix-a}
\lstinputlisting[language=Python, label={code:relation-dct}, style=simplecode, frame=single]{Code/relation_dct.py}

\end{document}
